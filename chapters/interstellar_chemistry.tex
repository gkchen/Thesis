The interstellar medium (ISM) is defined as the matter and radiation that exists between star systems and galaxies, the aggregated gasses form clouds with varying densities and sizes. Clouds of sufficient size and column density, may have regions of varying far-UV photon penetration. At the forefront, where the cloud is being bombarded by high energy UV photons capable of ionizing \ce{H}, the temperature reaches values of 1000's of Kelvin. This ionization front is almost exclusively populated by \ce{H+} with no trace of complex molecules. As the attenuation increases inside the cloud to the point where little to no UV radiation can penetrate, the temperature drops to $\sim 10$ K. The region is dominated by neutral molecules of various complexity called a cold molecular cloud. In between the ionization front and molecular cloud is called the photo-dissociation region (PDR), also called the photon-dominated region, that bridges the gap from atomic ions to neutral molecules. The region within the PDR where the temperatures are range from 100 K to 10 K are of particular interest, as these are where both ions (\ce{C+}, \ce{H+}, \ce{O+}, etc.) and molecules (\ce{H2}, \ce{O2}, \ce{CO}, etc.) coexist.\cite{Hollenbach1997}

Although the range of elements that make up the clouds is limited to primarily \ce{H}, \ce{C}, and \ce{O}, the breadth of molecules and ions that exist within these clouds is non-trivial. Studies of the Orion Bar show evidence of highly complex molecules including polycyclic aromatic hydrocarbons (PAHs) evidenced by emissions in the 3 micron range.\cite{Sloan1997,Bregman2002} These complex molecules are electronically excited by UV radiation, which then is emitted through rotational and vibrational transitions in the near IR. Of these emissions, an unknown, but prevalent 89.19 GHz line, called the X-ogen line) was observed in various regions of the sky.\cite{Buhl1970} Soon after, it was determined to be the molecular ion \ce{HCO+}, and a nearby line at 89.49 GHz was determined to be the isomer \ce{HOC+}.\cite{Gudeman1982} Colloquially, the combination of the two isomers is represented as \ce{[HCO+]} and called the formyl isomers. Both species were detected with varying strength in interstellar bodies with density ratios \ce{HCO+}/\ce{HOC+} ranging from 12408 in S140, to 50-120 in NGC 7023.\cite{Liszt2004} These wide variations have been a topic of great interest in the astrochemical community. One of the processes that may help explain the variations is that of \ce{C+ + H2O}, which produces both isomers at a branching ratio (\ce{HOC+}:\ce{HCO+}) unknown at cold temperatures, but interrogated at 305 K (86:14).\cite{Freeman1987} The goal described in this thesis is to build an apparatus that can determine the branching ratio and reaction rates at cold temperatures (10 K).