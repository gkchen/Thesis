To produce a beam of cold particles, various components need to be held within specific temperature ranges to ensure proper operation. Considering neon as the buffer gas species of choice, the experimental cell needs to be maintained at ~20K to prevent the neon from freezing to the walls and maintain a high stagnation density that allows for tuning of the flow regime. Conversely, we need the cryopumping shield surrounding the experimental cell to maintain a temperature $<17K$ so that the neon that escapes the cell is readily captured, as the turbo connected to the stem chamber cannot keep up with the gas load. A lack of proper cryopumping results in high densities in the chamber, which scatters the beam.

The PTR 40K cooling stage has 40W of cooling power, while the lowest 4K stage has only 4W. The low cooling power of the lowest stage means that extra care is needed to minimize the heat transfer to the stage from the higher temperature regions including black body radiation and conducted heat from high temperature surfaces.

Material choices used in the CBGB are dictated by their thermal conductivity down to the temperature ranges of interest. At room temperature, thermal conductivity ($k$) of a material is dominated by transfer of energy via phonons through the material. In this regime, different alloys and purities of a material do not greatly affect the conductivity. But once we enter cryogenic temperatures, the conductivity is dominated by electron motion through the material, meaning that purer samples have fewer imperfections to scatter off of, yielding higher conductivities. At these temperatures, it's convenient to characterize the conductivity of a material with the residual resistance ratio ($RRR=\frac{R(T=295K)}{R(T=4K)}$), which can be related to the thermal conductivity with the Wiedemann-Franz Law.\cite{White2009}

Al 6061 was chosen for the radiation shield for its thermal conductivity ($k_{Al6061}(T=40K) = 70$W/(m K)\cite{NIST}), ease of machining, as well as lightweight properties. The thermal mass of the aluminum shield coupled with its relatively lower thermal conductivity (compared to Cu 10100) means the cool down of this region limits the cool down process to ~6hr until at workable temperatures. The face of the aluminum shield coinciding with the outgoing buffer gas beam was fitted with a set of stacked chevron baffles as seen in figure \ref{fig: SW chamber}. The baffle design blocks stray light from entering the radiation shield, while enabling gas to pass from the enclosed shields into the stem region, preventing high density regions from forming and scattering the beam. Conversely, the baffles allow for gas within the stem region to reenter the cryogenic shields and facilitate cryopumping of stray particles.

The copper region contains the experimental cell, encompassed by a copper enclosure that acts as a cryopumping surface at the appropriate temperatures. Cu 10100, or oxygen free copper, was chosen for these components for its high thermal conductivity through to 4K, $RRR=2000$, $k_{Cu10100}(4K) = 10^4$W/(m K)\cite{NIST}. Because it is heat sunk into the same cooling stage as the experimental cell, the copper enclosure does not act as a radiation shield for it does not redirect the heat load away from the experimental cell's cooling surfaces. To maintain a beam, while still cryopumping, the experimental cell is held at a temperature higher than that of the cryopumping shield, the cell is connected to the shield with a SS316 stand off, which has poor thermal conductivity, especially at lower temperatures. A resistive heater and temperature sensor diode (DT-670) are mounted to the cell and controlled with a Lakeshore controller (Model 325) to maintain a set point temperature.

The main heat loads onto the system are those from the black body radiation, as well as the stainless steel rods supporting the shields from the top mounting plate. The heat load and transient solutions over the system may be determined by solving the heat/diffusion equation.

\begin{equation}
	\frac{\partial u(x, t)}{\partial t} = D \nabla^2 u(x, t)
\end{equation}

\begin{equation}
	\dot{Q} = \frac{A}{l}\int_{T_1}^{T_2} \lambda(T) dT
\end{equation}

\begin{equation}
	E = \sigma T^4
\end{equation}

\begin{equation}
	\lambda_{max}T \approx 2900\mu \text{m K}
\end{equation}

\begin{equation}
	\dot{Q} = \sigma A (T_1^4 - T_2^4)\frac{\epsilon_1 \epsilon_2}{\epsilon_1 + \epsilon_2 - \epsilon_1 \epsilon_2}
\end{equation}