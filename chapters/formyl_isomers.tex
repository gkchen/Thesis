In the interstellar medium, many reaction occur, in photon dominated regions (PDR's), as these are the regions where both atomic ions and molecules would coexist. At low temperatures, the rate constant of ion-dipole reactions increase, while other rates decrease or do not change, meaning ion-molecule reaction dominate in these cold (10 K - 100 K) PDR's. In particular, we are interested in the cold production of the formyl isomers (\ce{HCO+ / HOC+}) via \ce{C+} and \ce{H2O}.

\begin{align}
	\ce{C+ + H2O & -> HCO+ + H} \label{r: C+H2O->HCO} \\
	\ce{& -> HOC+ + H} \label{r: C+H2O->HOC} \\
	\ce{HCO+ + H2O & -> H3O+ + CO} \label{r: HCO+H2O->H3O} \\
	\ce{HOC+ + H2O & -> H3O+ + CO} \label{r: HOC+H2O->H3O}
\end{align}

Of which, we verify that the rates of reactions \ref{r: HCO+H2O->H3O} and \ref{r: HOC+H2O->H3O} are indistinguishable in figure \ref{fig: [HCO] rate} such that we may instead write:

\begin{equation}
	\ce{[HCO]+ + H2O -> H3O+ + CO} \label{r: [HCO]+H2O->H3O}
\end{equation}

Where \ce{[HCO]+} is used to represent both isomers. 

By definition, these formyl isomers of reactions \cref{r: C+H2O->HCO,r: C+H2O->HOC} have identical mass and thus, cannot be readily read off by the TOF system. To be able to separate the isomer products, we need to be able to separate their masses. By introducing a gas into the system with a proton affinity in between the isomer products, we may selectively react only one the less stable \ce{HOC+} isomer. This also yields a distinct $m/z$ peaks originating from separate isomers as seen in reactions \ref{r: X+HOC->XH} and \ref{r: X+HCO->NR}. But by using an external gas, we are doing an indirect measurement, and as such, it may add unintended complications. Certain gasses are more reactive and may react with the excited \ce{Be+}, \ce{C+}, or any other ionized species in the trap. Another possibility is that the \ce{COH+} may isomerize due to interactions with the gas as shown in reaction \ref{r: X+HOC->HCO}.\cite{Love1987}

\begin{align}
	\ce{HOC+ + X & -> XH+ + CO} \label{r: X+HOC->XH} \\
	\ce{& -> HCO+ + X} \label{r: X+HOC->HCO} \\
	\ce{HCO+ + X & -> no reaction} \label{r: X+HCO->NR}
\end{align}

\begin{table}[H]
	\centering
	\label{tab: affinities}
	\begin{tabular}{|l|c|}
	\hline
 & Affinity (kcal/mol)   \\
 \hline
	\ce{CO}* & 427               \\
	\ce{Kr}  & 425               \\
	\ce{HF}  & 490               \\
	\ce{N2}  & 495               \\
	\ce{Xe}  & 496               \\
	\ce{NO}  & 531               \\
	\ce{CO2} & 548               \\
	\ce{CH4} & 552               \\
	\ce{HCl} & 564               \\
	\ce{HBr} & 569               \\
	\ce{N2O} & 571               \\
	*\ce{CO} & 594 \\
	\hline
	\end{tabular}
	\caption{Proton affinities of gasses between formyl isomers where (*) indicates H bonding location.}
\end{table}