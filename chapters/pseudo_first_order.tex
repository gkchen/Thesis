When describing chemical reaction rates, it's instructive to use the order of reaction defined by the sum of the order constituents contribute to the reaction. For example, if we have a bimolecular reaction of:

\begin{equation*}
	\ce{[A] + [B] -> [C] + [D]}
\end{equation*}

Where \ce{[X]} is the concentration of species \ce{X} with stoichiometric coefficient 1. The appearance of products \ce{[C]} and \ce{[D]} is directly equivalent to the disappearance of \ce{[A]} and \ce{[B]}. Similarly, the rates at which \ce{[A]} and \ce{[B]} disappear are coupled so that $\frac{d \ce{[A]}}{dt} = \frac{d \ce{[B]}}{dt}$. The differential form of reactant \ce{[A]} can be written as:

\begin{equation*}
	\frac{d \ce{[A]}}{dt} = -k \ce{[A]}\ce{[B]}
\end{equation*}

Where $k$ is the rate constant. The solution to the differential form is then:

\begin{equation*}
	\frac{\ce{[A]}}{\ce{[B]}} = \frac{\ce{[A]0}}{\ce{[B]0}}e^{(\ce{[A]0 - [B]0}])kt}
	\label{eq: second order}
\end{equation*}

Where the subscript $0$ denotes the initial concentration of a reactant. Finding the characteristic decay time when $(\ce{[A]0 - [B]0})k\tau = -1$, the rate constant is $k=\Gamma/(\ce{[A]0 - [B]0})$, where $\Gamma=1/\tau$. This is experimentally troublesome because both reactants need to be simultaneously measured to yield a rate constant.

In our experiment, we are only trapping a few ions (~1000 max) in the trap while flooding the chamber with neutral reactants (~$1 \times 10^11$) from either a cryogenic buffer gas beam or a leak valve. In either case, the concentration of one reactant is held effectively fixed, while the other is depleted. Letting \ce{[A]} and \ce{[B]} be the concentrations of the scarce and flooded reactants respectively: $\ce{[B]0} \gg \ce{[A]0}$, and $\ce{[B]0} \approx \ce{[B]}$. This simplifies equation \ref{eq: second order} to a pseudo-first-order rate equation.

\begin{equation*}
	\ce{[A]} = \ce{[A]0}e^{-k\ce{[B]}t}
\end{equation*}

We can readily identify the rate constant $k=\Gamma/\ce{[B]}$, with dimensions cm$^3$/s. The reactions discussed in this thesis are exclusively of the pseudo-first-order.

