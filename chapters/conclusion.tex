The experiment has broken new ground in ion-molecule reactions at various reaction temperatures, but there is still much left unexplored. The recent inclusion of two Lioptec dye lasers will allow for probing of internal state distributions of the \ce{H2O} in the CBGB via REMPI. Further experiments may go towards the inclusion of polycyclic aromatic hydrocarbons (PAH's) in the CBGB, where the dye lasers would also be able to directly detect the production and cooling of these species within the beam.

An avenue that was discussed but not extensively explored is the possibility of looking at stereodynamics between moelcules and ions with the apparatus. A key feature is that the molecules approaching the ions in the trap come in a predetermined direction, if the excited \ce{Be+} P-state orbital is oriented in a particular direction relative to the oncoming molecules, we may find that there is a change in reaction rates or even branching ratios.

The possibilities are endless.