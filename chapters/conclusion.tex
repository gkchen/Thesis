The experiment has broken new ground in ion-molecule reactions at various reaction temperatures. We have found that although we may calculate the collision rate of ions and molecules considering their long range interaction potentials, short range dynamics can suppress nominally exothermic channels. Yet, the extrapolation of the role of dynamics into isotopically selective bond breaking does not hold. Lastly, we demonstrated the effectiveness of the platform for producing and observing reactions between ions and molecules of astrochemical importance at various reaction temperatures, ranging from 10 K to 300 K.

Despite the breadth of work presented here, there is still even more science left unexplored. The recent inclusion of two Lioptec dye lasers will allow for probing of internal state distributions of the \ce{H2O} in the CBGB via REMPI. More ambitious experiments may go towards the inclusion of polycyclic aromatic hydrocarbons (PAH's) in the CBGB to look at truly complex hydrocarbon chemistry at cryogenic temperatures.