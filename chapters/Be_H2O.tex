In loading both Be$^+$ and C$^+$ in the trap and exposing them to H$_2$O from the leak valve, we find that the two proceed at drastically different rates, which does not make sense looking at capture rates. We investigate the rate of Be$^+$ + H$_2$O since C$^+$ + H$_2$O proceeds at the capture limit.

\section{Copied From Paper}
\todo[inline]{Edit This}

\subsection{Abstract}
We investigate reactions between laser-cooled Be$^+$ ions and room-temperature water molecules using an integrated ion trap and high-resolution time-of-flight mass spectrometer. This system allows simultaneous measurement of individual reaction rates that are resolved by reaction product. The rate coefficient of the Be$^+$($^2\text{S}_{1/2}$) + H$_2$O $\rightarrow$ BeOH$^+$ + H reaction is measured for the first time and is found to be approximatedly two times smaller than predicted by an ion-dipole capture model. Zero-point-corrected quasi-classical trajectory calculations on a highly accurate potential energy surface for the ground electronic state receal that the reaction is capture-dominated, but a submerged barrier in the product channel lowers the reactivity. Furthermore, laser excitation of the ions form the $^2\text{S}_{1/2}$ ground state to the $^2\text{P}_{3/2}$ state opens new reaction channels, and we report the rate and branching ratio of the Be$^+(^2\text{P}_{3/2})$ + H$_2$O $\rightarrow$ BeOH$^+$ + H and H$_2$O$^+$ + Be reactions. The excited-state reactions are nonadiabatic in nature.

\subsubsection{Introduction}

Low-temperature reactions of simple ions with small molecules play a central role in astrochemical environments from interstellar clouds to cometary comae to planetary atmospheres, including that of Earth\todo{1,2}. The chemical evolution of interstellar molecular clouds ultimately yields the seedbed from which new stars and planets are born and the raw materials from which life likely developed. A firm understanding of the reaction rates for a host of elementary ion-molecule reactions is essential to accurately model these environments these environments. Techniques such as selected ion flow tubes (SIFTs)\todo{3}, guided ion beams\todo{4}, and supersonic flows (CRESU)\todo{5} have improved our empirical understanding of these processes; however, each has its own limitations.\todo{6,7} Theoretically, it has long been recognized that these ion-molecule reactions are often barrierless, and their rates are frequenctly described by capture models.\todo{8} However, recent studeis have revealed taht dynamical features can sometimes prevail,\todo{9-11} in chich case statistical reatments may not be accurate.\todo{12,13} Therefore, new experimental and theoretical efforts are needed to accurately address ion-molecule chemistry.

We have developed an approach, adapted from the ultracold ion community,\todo{14,16} to study reactions of atomic ions with small molecules. Here we report the use of this approach to study the reaction of Be$^+$ with gas=phase water for the first time. There have been very few experimental studies of gas-phase reactions between metal ions and water, expecially at low temperature, despite their importance for metal ion chemistry in a range of environments.\todo{17-19}

Singly ionized beryllium is a paricularly attractive metallic reactant to use for such studies because it is both theoretically tractable and experimentally highly controllable. The relatively simple electronic structure of theis three-electron ion allows both highly accurate characterization of its electronic structure and laser cooling,\todo{20}