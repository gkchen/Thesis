This thesis chronicles the experimental work done to realize an apparatus for cold ion-molecule chemistry of species of astrochemical interest, and attempts to understand chemistry in general along the way. The overarching goal of my research has been the understanding of the \ce{HOC+}:\ce{HCO+} branching ratio from the \ce{C+ + H2O} reaction at low temperatures described in Chapter \ref{sec: [HCO]}.

On our path to understanding of the \ce{C+ + H2O} reaction network, we first examine the reactions of \ce{Be+} with \ce{H2O} and \ce{HOD}. Beryllium cation chemistry is not well understood. We find that promoting the \ce{Be+} from the ground \ce{^2S1/2} ground state to the \ce{^2P3/2} excited state opens up new reaction pathways when reacting with \ce{H2O}. Although nominally exothermic, \ce{Be+(^2S1/2) + H2O} does not proceed as predicted with capture theory. Simulations performed by Hua Guo's group show that a submerged barrier causes dynamics to suppress the rate constant, which is eliminated when promoted to the excited state. Furthermore, we explore bond-selective chemistry by replacing \ce{H2O} with \ce{HOD} to find that despite the large role dynamics played in the rate constant, the H and D bond breaking is consistent with statistical theory.

Learning from the previous experiments, we investigate the \ce{C+ + H2O} reaction and product branching ratios at cryogenic temperatures. Due to the fact that the isomer products have the same charge to mass ratio, a secondary reaction is needed to separate the signals. Great lengths were taken to experimentally verify the secondary reactions that occur while the reaction products are still exposed to the beam, as well as ensure there aren't any long lived internally excited states that would skew the observed branching ratio. Our results show a deviation from the previously determined ratio found at a reaction temperature of 300 K, in line with the theory provided by Hua Guo.