Our experimental results yield directly measured branching ratios $(58\pm5):(42\pm5)$ and $(58\pm1):(42\pm1)$ via titration with \ce{CO2} and \ce{^15N2}, in good agreement with one another. As the results from the \ce{^15N2} titration were without unknown peaks, we take that to be the proper experimental result. Hua Guo provided theoretical support in performing QCT calculations on the \ce{C+ + H2O} reaction dynamics at collision temperatures around 10 K. They find an initial branching ratio of 97:3, with 19\% of their \ce{HCO+} above the self-isomerization barrier, this brings their ratio down to 74:26. Hua also provided us with calculations on the possible percentage of isomerization due to reaction \ref{r: X+HOC->HCO} where \ce{X -> ^15N2} to be 17\%, consistent with our experimental results. Adjusting our experimentally determined ratio by the possible isomerization, we yield a ratio result of $(70\pm1):(30\pm1)$, this is much closer to the QCT results.

These results are in contrast to the room temperature result of 86:14\cite{Love1987}, but in better agreement with phase space calculations claiming a branching ratio of 67:33 in the literature.\cite{DeFrees1984}