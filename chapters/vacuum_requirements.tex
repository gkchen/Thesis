To reliably laser cool and trap ions into crystals, it is ideal to have ultra-high vacuum (UHV) which is generally defined as having a pressure $<10^{-9}$ Torr. The characteristic collision rate between a monopole and polarizable neutral is defined by the $C_4/r^4$ attractive term and is called the Langevin collision rate. In a vacuum chamber, we tend to find the predominant gas left after baking is \ce{H2}, which will have collisions with the trapped ions at the Langevin rate.

By knowing the rate of

\begin{equation}
	\ce{Be+(^2P3/2) + H2 -> BeH+ + H}
	\label{eq: Be+H2->BeH}
\end{equation}

to be \todo{rate}\cite{Roth2006}. With an ion gauge, we find our vacuum to be ~1e-10 Torr, and verified via \ce{Be+} fluorescence decay due to reactions with background \ce{H2}.

We want to make sure that there is as little \ce{H2O} in the chamber as possible to ensure that the data we take with the water from the CBGB is exclusively from the CBGB and not due to background water collisions.