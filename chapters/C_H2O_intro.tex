The interstellar medium (ISM) is defined as the matter and radiation that exists between star systems and galaxies, the aggregated gasses form clouds with varying densities and sizes. At high enough densities/sizes, these clouds attenuate the oncoming radiation from nearby light sources to where it may become opaque as viewed from the opposite side. These dark/dense clouds then have regions ranging from 1000's of Kelvin to 10's of Kelvin where the hottest regions are being bombarded by UV radiation consisting of almost exclusively ions \ce{H+}, \ce{C+}, etc. (HII regions), while the coldest regions are deeper within and predominately populated by neutral molecules (molecular cloud).\todo{cite these things}

In between these extremes, we have a mixing of both ions and molecules at low temperatures ranging from 10 K to 100 K.\todo{citation} In this regime, with an abundance of both ions and molecules, the reactions with ion-dipole interactions will dominate, as the rate constant increases as temperature decreases, unlike neutral-neutral and ion-induced dipole reactions. Within these clouds, one of the most prevalent molecular ions observed is that of the formyl group (or aldehyde group) \ce{HCO+} and its isomer \ce{HOC+}, represented together as \ce{[HCO]+}. These isomers have been observed in diffuse clouds with ratios (\ce{HCO+}:\ce{HOC+}) ranging from $\approx100:1$ to $>12000:1$. These discrepencies may arise due to the dominant avenues in which the formyl isomers can be produced, isomerization with \ce{H2} molecules, etc. One of the reaction networks that produce the isomers is of particular interest, \ce{C+ + H2O}.

\begin{align}
	\ce{C+ + H2O & -> HCO+ + H} \label{r: C+H2O->HCO} \\
	\ce{& -> HOC+ + H} \label{r: C+H2O->HOC}
\end{align}

Where the branching ratio of interest is defined to be the percentage \ce{HOC+} production to that of \ce{HCO+}, \ce{HOC+}:\ce{HCO+}. This ratio has been experimentally found to be 86:14 at room temperature\cite{Love1987}, but unknown at temperatures relevant to the ISM, where these kind of ion-dipole reactions would dominate.

By definition, these formyl isomers of reactions \cref{r: C+H2O->HCO,r: C+H2O->HOC} have identical mass and thus, cannot be readily read off by the TOF system. To be able to separate the isomer products, we need to be able to separate their masses. By introducing a gas into the system with a proton affinity in between the isomer products, we may selectively react only the less stable \ce{HOC+} isomer. This also yields a distinct $m/z$ peaks originating from separate isomers as seen in reactions \ref{r: X+HOC->XH} and \ref{r: X+HCO->NR}. But by using an external gas, we are doing an indirect measurement, and as such, it may add unintended complications. Certain gasses are more reactive and may react with the excited \ce{Be+}, \ce{C+}, or any other ionized species in the trap. Another possibility is that the \ce{HOC+} may isomerize due to interactions with the gas as shown in reaction \ref{r: X+HOC->HCO}.\cite{Love1987}

\begin{align}
\ce{HOC+ + X & -> XH+ + CO} \label{r: X+HOC->XH} \\
\ce{& -> HCO+ + X} \label{r: X+HOC->HCO} \\
\ce{HCO+ + X & -> no reaction} \label{r: X+HCO->NR}
\end{align}

\begin{table}[H]
	\centering
	\label{tab: affinities}
	\begin{tabular}{|l|c|c|}
		\hline
		& Affinity (kcal/mol) & Affinities (eV)   \\
		\hline
		\ce{O2}  & 422 & 18.3 \\
		\ce{H2}  & 424 & 18.4 \\
		\ce{Kr}  & 425 & 18.4 \\
		\ce{CO}* & 427 & 18.5 \\
		\ce{Kr}  & 425 & 18.4 \\
		\ce{HF}  & 490 & 21.2 \\
		\ce{N2}  & 495 & 21.5 \\
		\ce{Xe}  & 496 & 21.5 \\
		\ce{NO}  & 531 & 23.0 \\
		\ce{CO2} & 548 & 23.8 \\
		\ce{CH4} & 552 & 23.9 \\
		\ce{HCl} & 564 & 24.5 \\
		\ce{HBr} & 569 & 24.7 \\
		\ce{N2O} & 571 & 24.8 \\
		*\ce{CO} & 594 & 25.8 \\
		\hline
	\end{tabular}
	\caption{Proton affinities of gasses between formyl isomers where (*) indicates H bonding location.}
\end{table}