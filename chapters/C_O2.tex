Looking at the energy difference between the two isomers, we see that 

We want to figure out if the \ce{[HCO]+} isomers are still in their internally excited states, as that would change their reactivity with respect to the introduced titration gas \ce{X}.

By using \ce{O2}, we can see in table \ref{tab: affinities}, that we are only \todo{value} eV away from being able to react with the more stable \ce{HCO+}. Given that the main reaction (\ref{r: C+H2O->HCO}) is exothermic by 5 eV, if the molecule does not relax within the characteristic time of a collision, we may expect to see some proton exchange occur.

To start, we consider the reactions of \ce{C+} with \ce{O2}:

\begin{align}
	\ce{C+ + O2 & -> CO+} \label{r: C+O2->CO} \\
	\ce{& -> O+ + CO} \label{r: C+O2->O}
\end{align}

Where literature tells us that \ref{r: C+O2->CO} \ref{r: C+O2->O} both proceed at approximately $4 \times 10^{-10}$ cm$^{-3}$/s.

We introduce \ce{O2} into the chamber with \ce{Be+} and \ce{C+}