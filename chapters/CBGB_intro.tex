To reach reaction temperatures in the regime of 10K from a beam of molecules with trapped ions, a cryogenic buffer gas beam (CBGB) of neon with entrained water is employed. Numerous methods of creating cold beams of molecules exist, from Zeeman decelerators \cite{Narevicius2008}, Stark decelerators, to cryogenic buffer gas beams (CBGB). CBGB's in particular have the benefit of being species agnostic, where the resultant beam properties are not dependent on the target species at hand, rather, the buffer gas species. \todo{Add citations for decelerators}

By holding a cell filled with a noble gas above its vapor pressure, a volume of gas can be held at cryogenic temperatures. Other species of molecules or atoms may be introduced into the buffer gas cell via ablation, fill line, etc. The target species particles are then sympathetically cooled via collisions with the cold buffer gas. An aperture at one end of the cell allows for the extraction of the buffer gas and entrained target species into a ballistic beam. Holding the buffer gas cell temperature to above 17K for neon, and 4K for helium, in high vacuum allows us to accumulate an appreciable stagnation number density within the cell to produce a beam. \todo{Add graphic to show what a buffer gas beam is}

The Reynolds number is typically used to characterize the flow regime of the buffer gas beam. At the aperture of the buffer gas cell, the Reynolds number can be written as:

\begin{align}
	Re & \approx \frac{2 d_{aperture}}{\lambda} \nonumber \\
	& \approx \frac{8\sqrt{2} \dot{N} \sigma}{d_{aperture} \bar{v}} \label{eq: reynolds}
\end{align}

Where $d_{aperture}$ is the diameter of the aperture and $\lambda$ is the mean free path of the buffer gas particles.\cite{Hutzler2012} When the Reynolds number is low, $Re<1$, we find that there are on average $>1$ collisions at the aperture, meaning the particles escape with little to no interactions with other particles and is called the effusive regime. At high Reynolds numbers, $Re>100$, in the supersonic regime, there are many collisions and forward velocity boosting as well as internal velocity distribution narrowing occurs. In between, we find the intermediate regime, where we observe the onset of hydrodynamic entrainment of target species with mild forward velocity boosting. In all cases, the gasses inside the cell at thermal equilibrium follow the Maxwell-Boltzmann distribution.

\begin{equation}
	f(v) = \left(\frac{m}{2 \pi k T}\right)^{3/2}4 \pi v^2 e^{-\frac{m v^2}{2 k T}} \label{eq: mb_distribution}
\end{equation}

Where the mean velocity is:

\begin{equation}
	\bar{v} = \sqrt{\frac{8 k_B T}{\pi m}} \label{eq: mb_mean}
\end{equation}