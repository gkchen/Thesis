Under the understanding that the reactions of interest for this work follow a pseudo-first-order model, a theoretical framework is needed to compare our findings. To figure out the characteristic rate constant $k$ of a reaction, we want to model the interaction between the reactants, whether it be neutral-neutral, to monopole-dipole. To do this, we consider adiabatic capture theory, a study of the long range potentials between particles. A caveat is that the adiabatic capture theory is long ranged, only finding the rate at which a collision will occur, not necessarily when a reaction will happen. The probability of a reaction occurring requires modeling of short range interactions within the reaction complex, but understanding capture theory will yield the maximally allowed rate of reactions, if all collisions lead to a reaction.