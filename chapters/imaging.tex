Using the 313 nm laser, we fluoresce the \ce{Be+} ions and cool them down to a cloud or crystal in the ion trap. The scattered light is observed via our imaging system shown in figure \ref{fig: imaging system}. The components include the Andor iXon3 camera with EM gain, a 313 nm band pass filter, angled mirror, enclosing lens tubes, and Sill opjective lens with 0.2 NA, and 40 mm working distance. The alignment of our objective lens to camera imaging plane yields a magnification of about $\times5.5$.

\begin{figure}[H]
	\centering
	\includegraphics[width=0.8\textwidth]{images/imaging_system.jpg}
	\caption{The iXon3 camera is mounted onto a 3 axis translation stage as well as the enclosed imaging pathway. The imaging tubes include the Sill objective lens at the end, inserted into the reentrant flange, an angled mirror, and a 313 nm bandpass filter placed at the camera input.}
	\label{fig: imaging system}
\end{figure}

All of the imaging components are rigidly mounted onto the 3 axis translation stage allowing us to move the focal point without changing the magnification.

\begin{figure}[H]
	\missingfigure{image of cloud}
\end{figure}

Considering all the components, the total efficiency of our imaging system is:

\begin{equation}
	\epsilon = \Omega \alpha \beta \gamma
	\label{eq: fluorescence efficiency}
\end{equation}

Where $\Omega$ is the solid angle the reentrant objective appends, $\alpha$ is the camera's quantum efficiency at 313 nm, $\beta$ is the camera's exposure time, and $\gamma$ is the camera's gain. For a fluorescing ion scattering at $\Gamma \times (\rho_{pp} \approx 0.20)$, we expect on the order of $10^5$ counts per ion including the imaging inefficiencies.