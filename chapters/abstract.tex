This thesis details the development of a new platform for the interrogation of ion-molecule chemistry at cryogenic temperatures to experimentally observe reaction rates and branching ratios of fundamental reactions in the interstellar medium. By combining cryogenic buffer gas cooling, laser-cooled ion sympathetic cooling, and integrated mass spectrometry in an RF Paul trap. Cold molecular species produced in a cryogenic buffer gas beam react with trapped \ce{Be+} and \ce{C+} ions. Since charged reaction products are also trapped, ion imaging and time of flight mass spectrometry are used to study the reaction rates and identify the products.

I first describe the design and calibration of the apparatus from the cryogenic buffer gas beam, to the time of flight mass spectrometer. Then I will discuss the work done towards understanding quantum state resolved \ce{Be+} ion chemistry with \ce{H2O}. We find that when \ce{Be+} is in the ground state, a submerged barrier in the reaction entrance channel prevents about half of the incoming trajectories from reaching the nominally exothermic product channel with good agreement between theory and experiment. Next, I will discuss the introduction of \ce{HOD} to determine if there are similar dynamics involved in preferential bond breaking. Coupled with theory, our experiment does not distinguish between dynamical processes and statistical theory. Finally, I will describe the experimental results in determining the isomer branching ratio in the \ce{C+ + H2O} reaction at collision temperatures around 10 K, 100 K, as well as 300 K.