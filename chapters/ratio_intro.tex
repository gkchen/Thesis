Previous literature utilized gasses such as \ce{NO}, \ce{CH4}, \ce{N2O}, and \ce{Kr} to separate the isomers.\todo{citations} Of those \ce{Kr}, and \ce{Xe} are inert and would not react with any other ions in our trap, but are too heavy to reliably hold after a reaction. Of the others, \ce{NO} is caustic and will ruin the vacuum chamber if introduced, and thus was avoided. Attempts were make with \ce{N2O} and well as \ce{CH4}, but both had their own unique complications. \ce{N2O} rapidly reacts with \ce{Be+} and made reliable TOF traces unattainable due to the loss of the coolant ion. \ce{CH4} readily reacted with most of the ions in the trap to produce a multitude of mass peaks, greatly complicating the analysis with secondary, tertiary, and higher order reactions. In this section, I describe the methods and results using \ce{CO2} and \ce{^15N2} gasses to separate the isomer mass signatures.